\documentclass[showtrims, a4paper]{article}

\usepackage{hyperref}
\usepackage{nopageno}
\usepackage{xcolor}

\title{Küchendienst}
\date{}
\begin{document}
\maketitle{}

\begin{center}
\textbf{Frühstück}
\begin{itemize}
 \item Lieferung kommt zwischen 7:15 und 7:20, einfach mit reinnehmen
 \item Alle bauen mit auf und ab
 \item Müll, Kaffee und Tee müssen SPÄTESTENS 9:15 draußen stehen
\end{itemize}
\end{center}

\vspace{0.0cm}
\begin{center}
\textbf{Welches Haus hat Küchendienst?}

\vspace{0.1cm}
\begin{tabular}{l|l|l|l}
 Samstag 11:45 & Samstag 17:45 & Sonntag 11:45 & Sonntag 17:45 \\
 Haus 1 & Haus 2 & Haus 3 & Haus 4
\end{tabular}
\end{center}

\vspace{0.0cm}
\begin{center}
\textbf{Was gehört mittags alles zum Küchendienst?}
\begin{itemize}
 \item Seitlich am Kühlschrank nachlesen, was es es zu essen gibt
 \item Lieferung reintragen (kommt zwischen 11:30 und 12:00)
 \item Ein sinnvolles, von zwei Seiten, nutzbares Buffet aufbauen und beschriften
 \begin{itemize}
   \item Teller vor Essen, Saucen hinter Essen, \dots
 \end{itemize}
 \item Regelmäßig leeres Essen nachfüllen
 \item übriges Essen
   \begin{itemize}
     \item ENTWEDER zurück in die Container
     \item ODER in einem verschlossenen, mit Datum beschrifteten Container maximal einen Tag aufheben
   \end{itemize}
  \item Vollständige Container (also mit allen Tablets etc) nach draußen stellen;
    Abholung kann ab 13 Uhr ankommen
 \item Spühlmaschinen ein- und ausräumen, bis alles wieder sauber ist; Teller mit Öffnung zur Mitte ausrichten
\end{itemize}
\end{center}

\vspace{0.0cm}
\begin{center}
\textbf{Was ist abends anders?}
\begin{itemize}
 \item Zusätzlich den Kühlschrank auf- und ausräumen
 \item Von allem nur eine Packung öffnen und in Boxen umpacken
 \item Auf CdErnhof und Steinsgebisshof \emph{nicht} Allergikeressen (Zöliakie) aufbauen, um es nicht mit Gluten/Weizen zu verunreinigen
\end{itemize}
\end{center}
\end{document}
