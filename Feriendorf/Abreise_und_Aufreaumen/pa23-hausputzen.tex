% Compile with LuaLaTeX!
%
\documentclass[14pt,parskip=full+]{scrartcl}

% LaTeX packages
%\usepackage{fontspec}
\usepackage[ngerman]{babel}
\usepackage{graphics}
\usepackage{amsmath}
\usepackage{enumitem}
\setlist{noitemsep}
\usepackage{scrlayer-scrpage}
\usepackage{longtable}
\usepackage{graphicx}
\usepackage[table]{xcolor}
\usepackage{colortbl}
\usepackage{microtype}
\usepackage{calc}
\usepackage{geometry}
\usepackage{tabularx}
\usepackage{tikz}
\usetikzlibrary{calc}
\usepackage[
    colorlinks=True,
    urlcolor=black!70!white,
    pdfauthor={PfingstAka19-Orgas},
    unicode
]{hyperref}

% Load Fonts and define colors
\usepackage{libertine}

% Style definitions
\setlength\intextsep{0pt}
\addtokomafont{disposition}{\rmfamily}
\arrayrulecolor{design2}
\geometry{top=1.27cm,bottom=1.27cm,left=1.27cm,right=1.27cm}
\newcommand{\head}[1]{\textcolor{white}{\textbf{#1}}}
\ihead[]{}
\cfoot[]{}
\ifoot[]{Offizieller Hinweis der Orgas}
\ofoot[]{}

\makeatletter
\newcommand{\gettikzxy}[3]{%
  \tikz@scan@one@point\pgfutil@firstofone#1\relax
  \edef#2{\the\pgf@x}%
  \edef#3{\the\pgf@y}%
}
\makeatother

\newcommand{\checkfield}{~ \hfill ~\raisebox{-5pt}{\tikz\node[draw,minimum width=16.8pt,minimum height=16.8pt]{};}}
\newcommand{\countfield}{~ \hfill ~\raisebox{-5pt}{\tikz\node[draw,minimum width=33.6pt,minimum height=16.8pt]{};}}
\newcommand{\pfeil}{~\ensuremath{\rightarrow}~}
\newcommand{\mysection}[1]{\vspace{-.7\baselineskip}\subsubsection*{#1} \vspace{-.7\baselineskip}}

\begin{document}

\parbox[t]{.8\textwidth}{\vskip0pt
\section*{Häuser aufräumen -- wie gehts?}
\subsection*{Sehr harte\footnotemark~ Frist: 9:50 Uhr!}
}
\footnotetext{wirklich hart!}
\parbox[t]{.2\textwidth}{\vskip0pt 
%\includegraphics[width=.18\textwidth]{PA19_logo2_grey.pdf}% TODO adapt path to logo
}

\bigskip

\def\arraystretch{1.3}

\mysection{Bitte stapeln und abhaken:}
\begin{tabularx}{\textwidth}{Xc}
Matratzen \pfeil eine aufs untere Stockbett, der Rest oben stapeln & \checkfield \\
Bettdecken \pfeil unten in das Stockbett stapeln & \checkfield \\
Kissen \pfeil unten in das Stockbett & \checkfield
\end{tabularx}


\mysection{Bitte zählen:}
\begin{tabularx}{\textwidth}{Xc}
Anzahl Matratzen (hoffentlich 13) & \countfield \\
Anzahl Bettdecken (13 sind ideal\footnote{überschüssige Decken und Kissen bitte ins Orgabüro, ebenso ausgeliehene Isomatten}) & \countfield \\
Anzahl Kissen (13 sind ideal\footnotemark[2]) & \countfield \\
Anzahl Stühle (7 sind ideal\footnote{Bei Bedarf überschüssige Stühle auf dem Hof verteilen oder holen}) & \countfield
\end{tabularx}

\mysection{Bitte aufräumen und saubermachen:}
\begin{tabularx}{\textwidth}{Xc}
Bettwäsche \pfeil in den Wagen im Hof & \checkfield \\
Gepäck \pfeil zur Pelikanhalle & \checkfield \\
Mülleimer \pfeil Müll in Tonnen neben Speisesaal trennen & \checkfield \\
Haus von oben nach unten kehren\footnote{Große Besen sind im Speisesaal, Handbesen und Schaufel im Bad} & \checkfield \\
Schränke leer? \pfeil Falls sich kein Besitzer findet, zur Fundsachenkiste am Orgabüro & \checkfield
\end{tabularx}
 
\textbf{Parallel: Hof \& Speisesaal aufräumen}

(siehe Zettel an Speisesaaltür)

\subsection*{Und jetzt zum Plenum!}
\vspace{-\baselineskip}
(Gepäck mitnehmen!)

\end{document}
