% Compile with LuaLaTeX!
%
\documentclass[14pt,parskip=full+]{scrartcl}

% LaTeX packages
%\usepackage{fontspec}
\usepackage[ngerman]{babel}
\usepackage{graphics}
\usepackage{amsmath}
\usepackage{enumitem}
\setlist{noitemsep}
\usepackage{scrlayer-scrpage}
\usepackage{longtable}
\usepackage{graphicx}
\usepackage[table]{xcolor}
\usepackage{colortbl}
\usepackage{microtype}
\usepackage{calc}
\usepackage{geometry}
\usepackage{tabularx}
\usepackage{tikz}
\usepackage{ulem}
\usetikzlibrary{calc}
\usepackage[
    colorlinks=True,
    urlcolor=black!70!white,
    pdfauthor={PfingstAka19-Orgas},
    unicode
]{hyperref}

% Load Fonts and define colors
\usepackage{libertine}
\definecolor{design}{HTML}{2f2f2f}
\definecolor{design2}{HTML}{3f3f3f}

% Style definitions
\setlength\intextsep{0pt}
\addtokomafont{disposition}{\rmfamily}
\arrayrulecolor{design2}
\geometry{top=1.27cm,bottom=1.27cm,left=1.27cm,right=1.27cm}
\newcommand{\head}[1]{\textcolor{white}{\textbf{#1}}}
\ihead[]{}
\cfoot[]{}
\ifoot[]{Offizieller Hinweis der Orgas}
\ofoot[]{}

\makeatletter
\newcommand{\gettikzxy}[3]{%
  \tikz@scan@one@point\pgfutil@firstofone#1\relax
  \edef#2{\the\pgf@x}%
  \edef#3{\the\pgf@y}%
}
\makeatother

\newcommand{\checkfield}{~ \hfill ~\raisebox{-5pt}{\tikz\node[draw,minimum width=16.8pt,minimum height=16.8pt]{};}}
\newcommand{\countfield}{~ \hfill ~\raisebox{-5pt}{\tikz\node[draw,minimum width=33.6pt,minimum height=16.8pt]{};}}
\newcommand{\pfeil}{~\ensuremath{\rightarrow}~}
\newcommand{\mysection}[1]{\vspace{-.7\baselineskip}\subsubsection*{#1} \vspace{-.7\baselineskip}}

\begin{document}

\parbox[t]{.8\textwidth}{\vskip0pt
\section*{Speisesaal aufräumen -- wie gehts?}
\subsection*{Sehr harte\footnotemark~ Frist: 9:50 Uhr!}
}
\footnotetext{wirklich hart!}
%\parbox[t]{.2\textwidth}{\vskip0pt 
%\includegraphics[width=.18\textwidth]{PA19_logo2_grey.pdf}}% TODO adapt path to logo

\bigskip

\def\arraystretch{1.3}

\begin{tabularx}{\textwidth}{Xc}
 Kehren & \checkfield \\
 Kiosk-Inhalte zum Kiosk des CdErnhofs bringen & \checkfield \\
 Tische und Tresen abwischen & \checkfield \\
 Herrenlose geöffnete Flaschen ausleeren falls notwendig & \checkfield \\
 Leere Flaschen in Getränkekästen sortieren & \checkfield \\
 Getränkekästen rausstellen & \checkfield \\
 Tische in langen Tafeln anordnen, Stühle an der hinteren Wand stapeln (vgl. Bild des Feriendorfs)& \checkfield
\end{tabularx}
\mysection{Küche}
\begin{tabularx}{\textwidth}{Xc}
 \uline{Alles} Geschirr spülen (bzw. Spülmaschine befüllen und ausräumen) und sortieren & \checkfield \\ 
 Sortiertes Geschirr auf den Tresen stellen & \checkfield \\
 Mülleimer leeren (in die Tonnen vor dem Gebäude) & \checkfield \\
\end{tabularx}




\vspace{2cm}




\parbox[t]{.8\textwidth}{\vskip0pt
\section*{Allgemeines Aufräumen -- wie gehts?}
%\subsection*{Sehr harte\footnotemark~ Frist: 9:50 Uhr!}
}
%\footnotetext{wirklich hart!}
\parbox[t]{.2\textwidth}{\vskip0pt 
%\includegraphics[width=.18\textwidth]{PA19_logo2_grey.pdf}% TODO adapt path to logo
}

%\bigskip

\def\arraystretch{1.3}

\begin{tabularx}{\textwidth}{Xc}
 Herumliegende Gegenstände zur Fundsachenkiste am Orgabüro & \checkfield \\
 Herumstehende und (per Definition\footnote{leer sind angebrochene Flaschen ohne Inhalt oder ohne Beschriftung/Namen}) leere Flaschen in die Kästen räumen & \checkfield \\
 Beschriebenes Papier und anderen Müll in die Mülltonnen werfen & \checkfield \\
 Stifte, leeres Papier, Klebeband, etc. zum Orgabüro bringen & \checkfield \\
\end{tabularx}

\subsection*{Mehr Hilfe für die Orgas}
\begin{tabularx}{\textwidth}{X}
 Lichterketten an den Wegen vorsichtig aufwickeln und zum Orgabüro bringen \\
 Zettel von Fenstern, Wänden, etc. abhängen und in den Papiermüll werfen \\
\end{tabularx}


\end{document}
